\documentclass{article}
\usepackage[english]{babel}
\usepackage[utf8]{inputenc}
\usepackage{graphicx}
\usepackage[margin=2cm]{geometry}
\usepackage{subcaption}
\usepackage{amsmath}
\begin{document}

method described \cite{wiki} \\

$$ xy' + y = \ln x + 1 $$

first we solve the homogeneous equation $xy' + y = 0$: \\

$$ xy' + y = 0 $$
$$ \frac{dy}{y} = -\frac{dx}{x} $$
$$ \ln y = A - \ln x $$
$$ y = \frac{C}{x} $$ \\

now assume that $C = C(x)$ and substitute $y = \frac{C(x)}{x}$ into the original
equation (for more complex situations see \cite{wiki}): \\

$$ x(\frac{C'(x)}{x} - \frac{C(x)}{x^2}) + \frac{C(x)}{x} = \ln x + 1 $$
$$ C'(x) = \ln x + 1 $$ \\

now, integrating (the $\ln x$ term is integrated by parts): \\

$$ C(x) = x\ln x - \int \frac{x}{x} dx + x + \hat{C} $$
$$ C(x) = x\ln x + \hat{C} $$ \\

so getting back to our $y$ \\

$$ y = \ln x + \frac{\hat{C}}{x} $$


\begin{thebibliography}{9}
\bibitem{wiki}
\begin{verbatim}
  https://en.wikipedia.org/wiki/Variation_of_parameters
\end{verbatim}
\end{thebibliography}

\end{document}
